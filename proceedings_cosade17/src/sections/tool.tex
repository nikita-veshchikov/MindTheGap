\section{Leakage Detection Tool}\label{sec:tool}

Several tools that can help designers of cryptographic systems
were already suggested and discussed in literature.

SILK\footnote{\url{http://www.ulb.ac.be/di/dpalab/download/SILK_v0.1.zip}} 
presented in 2014~\cite{DBLP:conf/acsac/Veshchikov14}
can be used to generate simulated traces based on C++ code. Thus, it allows
to generate tracesets during the early stages of development, in order to test an implementation
against any attack. However, SILK works only with high level, C++  source code and 
can not take into account reordering or replacement (even removal) 
of instructions that is often used by compilers during
optimization. Also, this tool does not detect flaws in implementations, 
it only allows to easily generate simulated traces.

A tool based on formal verification 
was presented at EUROCRYPT in 2015~\cite{DBLP:conf/eurocrypt/BartheBDFGS15},
capable of detecting design flaws in masking schemes.
This tool can analyze programs written using the EasyCrypt framework and its language and
it requires the designer to transform the original implementation (e.g. assembly or C code)
to EasyCrypt. Unfortunately, errors could potentially be introduced during this process
and there is no guarantee that the program written using EasyCrypt will be equivalent
to the program in the original programming language. To the best of our knowledge, 
free automated tools that can transform C or assembly 
(or other languages that are often used for development of cryptographic software in embedded systems) 
programs to EasyCrypt do not exist.
Moreover, this tool is not opensource and thus can not be used by the developer community.

A simulation-based tool that can be used to analyze masking implementations
was presented at FSE in 2016~\cite{DBLP:conf/fse/Reparaz16}.
It can be used with software and hardware implementations
and it requires only high-level implementation source code, such as C language.
Due to this fact it can also be blind to re-arrangements of operations 
(which can lead to side-channel leakage) created by the compiler.
Until today, the source code of this tool also remains 
unavailable\footnote{We have contacted the author, there is intention to eventually publish the code.}.

\subsection{ASCOLD}\label{subsec:ascold}

In order to assess the security of implementations at the assembly level, we developed a tool called ASCOLD, standing for Assembly Code Leakage Detection tool.
The tool is written in \texttt{python} and the source code is available on our 
website\footnote{\url{https://github.com/nikita-veshchikov/ascold}}.
ASCOLD uses assembly code as its input in order to run a simulation while checking for potential
issues that can cause side-channel information leakage.
The tool is compatible only with assembly code 
(which can be used as is during the development or extracted from the compiled binary file). 
Thus it is possible to be sure that the executed code will be exactly the same as the code
which is analyzed. Otherwise, it becomes impossible to provide any guarantees
on the quality of the analysis, i.e. be sure that no additional issues are introduced during compilation.

The simulation run by the tool does not use an instance of an execution i.e.,
we do not use specific values in order to run the program.
ASCOLD starts a program in an initial state and propagates all changes
such as combinations of values, their modifications and replacements
of one value by another. More precisely, it keeps track of which shares
or combinations of shares are stored in each register (or memory cell).
During any arithmetic or logical operation, shares stored in different operands
are verified, specifically we check whether we combine different shares of the same family without randomizing beforehand. Note that not all combinations are hazardous, yet we opt for such a conservative approach in order to speed-up the verification process.

In the same way, we verify the implementation for the device-specific distance-based leakages 
for every arithmetic/logical operation, SRAM store or load instruction that is executed. Analytically, we verify whether
the previously stored value and the new value cause the overwrite effect~\ref{overwrite}. Similarly, our tool checks the load/store instructions for remnant effects discussed in Section \ref{mem_remnant}. In addition, it features the matrix $R$ of \emph{neighbours}, which represents registers that can leak while another register is used (neighbour leakage effect, Section \ref{mem_remnant}).
In order to bootstrap the whole simulation, the developer needs to provide
a configuration file. The configuration file is a simple text file that contains information about the initial
state of the system i.e., it describes which registers or addresses in memory contain different secret shares of sensitive values.
As the result of the simulation, ASCOLD prints out a line number and the rule that was violated
by the program.

ASCOLD works with the AVR family of microcontrollers, 
it implements the most common memory instructions such as load and store 
as well as a set of commonly used (in cryptography) instructions such as arithmetic operations
(\texttt{add}, \texttt{mul}, $\dots$) and logical operations (\texttt{and}, \texttt{eor}, \texttt{or}, $\dots$). The same core principles can be applied
in order to build a similar tool for a different instruction set or to add 
new AVR instructions supported by newer microcontrollers.

\textbf{Limitations} The current version of our tool
incorporates our findings which are based on the ATMega163, other models of microcontrollers
might have slightly different (even additional) issues that cause unintentional information leakage.
Among other things, leakage described in Section~\ref{mem_remnant} is more likely to be different (affecting different sets of registers) in other models of AVR microcontrollers.
ASCOLD does not take into account the effects of pipelining which might be an issue
in case of a microcontroller which can potentially handle two different shares of the same sensitive
value (at different stages of the pipeline) during the same clock cycle.
We did not implement all AVR instructions, most importantly
the current version of ASCOLD does not support loops.
However, we implemented the most commonly used instructions and new instructions/rules
can be added due to the tool's extensibility. The lack of jump instructions (loops)
can be disregarded via loop-unrolled implementations.