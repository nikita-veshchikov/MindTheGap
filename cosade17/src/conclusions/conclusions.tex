\section{Conclusions} \label{sec:conclusions}

This work investigated the hazards in software masking, 
suggested a verification tool and established a secure, 
1st-order masked Sbox implementation against 1st-order, univariate attacks. 
Still, several important questions for future work arise. 
We demonstrated that removing the ILA-breaching effects is feasible, 
yet identifying the best clearing mechanism and minimizing the overhead is a topic for further exploration. 
Similarly, the current work is limited in AVR ATMega163 and needs to be 
extended to different devices and platforms. Moreover, higher-order evaluation 
techniques are still nascent and in this work we did not focus on 1st-order, 
yet multivariate attacks such as those that exploit horizontality~\cite{DBLP:conf/ches/BattistelloCPZ16}. 
In addition, note that the ILA effects are observable throughout an implementation. 
Not only the cipher-related operations but any manipulation of shares during I/O, RNG routines \emph{etc.} 
can create hazards. Thus, there is need for effort towards a fully hardened implementation. 
Last but not least, we stress that the effects identified depend on the architecture and the physical layer, 
thus preventing them in the assembly layer is, in principle, less efficient and prone to errors. 
Ideally, we should strive towards custom-made microcontrollers that enforce ILA in hardware, 
without the addition of countermeasures such as threshold implementations~\cite{DBLP:conf/icics/NikovaRR06}.  
