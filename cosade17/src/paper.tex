%%%%%%%%%%%%%%%%%%%%%%% file template.tex %%%%%%%%%%%%%%%%%%%%%%%%%
%
% This is a general template file for the LaTeX package SVJour3
% for Springer journals.          Springer Heidelberg 2010/09/16
%
% Copy it to a new file with a new name and use it as the basis
% for your article. Delete % signs as needed.
%
% This template includes a few options for different layouts and
% content for various journals. Please consult a previous issue of
% your journal as needed.
%
%%%%%%%%%%%%%%%%%%%%%%%%%%%%%%%%%%%%%%%%%%%%%%%%%%%%%%%%%%%%%%%%%%%
%
% First comes an example EPS file -- just ignore it and
% proceed on the \documentclass line
% your LaTeX will extract the file if required
%
\RequirePackage{fix-cm}
%
%\documentclass{svjour3}                     % onecolumn (standard format)
%\documentclass[smallcondensed]{svjour3}     % onecolumn (ditto)
%\documentclass[smallextended]{svjour3}       % onecolumn (second format)
%\documentclass[twocolumn]{svjour3}          % twocolumn
%
\documentclass[runningheads, a4paper, 10pt]{llncs}

%\smartqed  % flush right qed marks, e.g. at end of proof
%
\usepackage{graphicx}
\usepackage{caption}
\usepackage{subcaption}
\usepackage{url}
\usepackage{amssymb}
\usepackage{amsmath}
\usepackage{booktabs}    
\usepackage{cite}    
\usepackage{comment}    
\usepackage{subfig}    
\usepackage{tabu}

%\usepackage[T1]{fontenc}
%\usepackage{libertine}%% Only as example for the romans/sans fonts
%\usepackage[scaled=0.85]{beramono}


\usepackage{float}
\hyphenation{en-co-ded}

%
% \usepackage{mathptmx}      % use Times fonts if available on your TeX system
%
% insert here the call for the packages your document requires
%\usepackage{latexsym}
% etc.
%
% please place your own definitions here and don't use \def but
% \newcommand{}{}
%
% Insert the name of "your journal" with
% \journalname{myjournal}
%
\begin{document}
\bibliographystyle{plain}


\title{Mind the Gap: Towards Secure 1st-order Masking in Software
\thanks{
Removed for submission
  % The work described in this paper has been supported 
  % by the Netherlands Organization for Scientific Research NWO under project ProFIL (628.001.007).
}}

%\titlerunning{Short form of title}        % if too long for running head

\author{Author list removed for submission}

\institute{Institute list removed for submission}

\maketitle

\begin{abstract}
Software-based cryptographic implementations are vulnerable to side-channel analysis. To protect against it, implementors often opt for masking countermeasures, which ensure theoretical protection against value-based leakages. However, the practical effectiveness of masking is halted by 
physical effects such as glitches and distance-based leakages, which violate the \emph{independent leakage assumption} (ILA) and result in security order reductions. This paper aims to address this gap between masking theory and practice in the following threefold manner. First, we perform an in-depth investigation of the device-specific effects that invalidate ILA in the AVR microcontroller ATMega163. Second, we provide an automated tool, capable of detecting ILA violations in AVR assembly code. Last, we craft the first (to our knowledge) \emph{``hardened"} 1st-order ISW-based, masked Sbox implementation, which is capable of resisting 1st-order, univariate side-channel attacks. Enforcing the ILA in the masked RECTANGLE Sbox requires $<>$ clock cycles, i.e. a $<>$-fold increase compared  to a naive 1st-order ISW-based implementation. 

 \keywords{Masking, AVR, independent leakage assumption, distance-based leakage, RECTANGLE, SCA }

\end{abstract}

\section{Introduction}
Nowadays, the explosive growth of the ``Internet of Things'' (IoT) is reshaping modern society, pervading its infrastructure and communications.
The rapid price drop in IoT components has transformed everyday products, enhancing them with network connectivity and information exchange capabilities. Amidst this new status quo, devices, ranging from cheap sensors to expensive vehicles, are required to maintain a heightened level of theoretical and physical security.

For instance, side-channel attacks (SCA) allow adversaries to recover sensitive data, 
by observing and analyzing the physical characteristics and emanations of a cryptographic 
implementation~\cite{DBLP:conf/crypto/KocherJJ99}. 
Such physical attacks motivated research towards countermeasures 
that perform noise amplification\todo{reviewer: ``towards countermeasures that perform \_a.o.\_ noise amplification''}, 
thus hindering the adversary's recovery capabilities. 
A common choice for provably secure, noise-amplifying software countermeasure is 
masking~\cite{DBLP:conf/crypto/ChariJRR99,DBLP:conf/crypto/IshaiSW03}. 
Masking employs secret-sharing techniques that establish theoretical security 
against the value-based leakage model. Rephrasing, masking secures implementations 
against adversaries that can only extract information about the value being 
processed at a given time. This underlying assumption is often referred to as the 
\emph{independent leakage assumption} (ILA)~\cite{DBLP:conf/eurocrypt/RenauldSVKF11}. 
Unfortunately, the exact values under manipulation are not always visible at a given 
layer of abstraction, e.g. at assembly code and such a limited adversarial model is 
not applicable in many practical, software-based scenarios. For instance, devices often 
exhibit \emph{distance-based leakages}, which can reduce the security of the masking 
countermeasure~\cite{DBLP:conf/cardis/BalaschGGRS14, DBLP:journals/iacr/GrootPPSB16}. 
Likewise, coupling effects~\cite{DBLP:conf/eurocrypt/RenauldSVKF11} and 
glitches~\cite{DBLP:conf/ches/MangardS06} can pose similar security hazards.

This work attempts to bridge the gap between theory and practice in the masking countermeasure with the following threefold contribution. First, we investigate several effects that violate ILA in an ATMega163 microcontroller and subsequently, we establish solutions that mitigate these issues. Second, we use this knowledge in order to build an assembly-oriented tool that is capable of detecting ILA violations in AVR-based masked implementations. Third, assisted by the developed tool, we craft the first (to our knowledge) 1st-order masked implementation in ATMega163 that is capable of resisting 1st-order, univariate attacks. In other words, we enforce the ILA in order to severely limit the informativeness of 1st-order leakages, forcing the adversary to resort to 2nd-order attacks. As a proof of concept, we develop a ``hardened" 1st-order, ISW-based~\cite{DBLP:conf/crypto/IshaiSW03}, bitsliced Sbox for the RECTANGLE cipher~\cite{DBLP:journals/chinaf/ZhangBLR0V15}. The ``hardened'' implementation requires 1319 clock cycles, a 15-fold increase compared to a ``naive" 1st-order, ISW-based, bitsliced Sbox of the same cipher. 

The rest of this paper is organized as follows. In Section~\ref{sec:background}, we provide preliminaries w.r.t. masking, the experimental setup and the evaluation techniques we employ. In Section~\ref{sec:ila_effects} we offer a detailed description of all the ILA-breaching effects that we have identified in ATMega163. 
Section~\ref{sec:tool} discusses the development of the assembly verification tool. Section~\ref{sec:rectangle} details the construction of a ``hardened'' RECTANGLE, 1st-order masked Sbox for ATMega163. We conclude and discuss future work in Section~\ref{sec:conclusions}.



\section{Background} \label{sec:background}
\subsection{Boolean Masking \& Order Reduction}
Chari et al., Goubin et al. and Messerges~\cite{DBLP:conf/crypto/ChariJRR99,DBLP:conf/ches/GoubinP99,DBLP:conf/fse/Messerges00} were among the first to suggest splitting intermediate
values with a secret sharing scheme, in order to force attackers to analyze higher-order statistical moments. Analytically, a $d$th-order Boolean masking scheme splits a sensitive value $x$ into $d+1$ shares $(x_0, x_1, \dots, x_d)$, as shown below.
\begin{equation}
x = x_0 \oplus  x_1 \oplus \dots \oplus x_d
\end{equation}
The shares $(x_0, x_1, \dots, x_d)$ are also referred to as the $(d+1)$-family of shares corresponding to $x$~\cite{DBLP:conf/ches/RivainP10}. Given that the ILA holds and assuming sufficient noise, it has been shown that the number of traces required for a successful attack grows exponentially w.r.t. the order $d$~\cite{DBLP:conf/crypto/ChariJRR99,DBLP:conf/eurocrypt/ProuffR13}. Several implementation options have been suggested for the masking countermeasure, ranging from lookup-table techniques~\cite{DBLP:conf/eurocrypt/Coron14,DBLP:conf/ctrsa/WangVGX15} to $GF$-based circuits~\cite{DBLP:conf/crypto/IshaiSW03,DBLP:conf/ches/RivainP10,cryptoeprint:2016:264}.

In parallel with the development of masked implementations, side-channel research focused on the practical evaluation of the countermeasure. Balasch et al.~\cite{DBLP:conf/cardis/BalaschGGRS14} put forward the concepts of value-based and distance-based leakages, as well as the notion of order reduction. We briefly restate their definitions below.\\\\
\textbf{Value/Distance-based leakage function}: A leakage function $L(.)$ consists of a deterministic part $L_d(.)$ and random additive noise $N$. The leakage function is \emph{value-based} if $L_d(.)$ can only take arguments from the set of intermediate values produced by the masking scheme. The leakage is \emph{distance-based} if $L_d(.)$ can  take arguments from the set that contains all possible pairwise combinations of intermediate values. The combination can imply operations such as XOR, concatenation, etc.  \\\\
\textbf{Order-reduction theorem}: A $d$th-order secure masking scheme under value-based leakages is $\lfloor \frac{d}{2} \rfloor$th-order secure under distance-based leakages.\\\\
The order-reduction theorem has been verified experimentally for orders $d=1,2$ by Balasch et al.~\cite{DBLP:conf/cardis/BalaschGGRS14} in AVR-based and 8051-based devices. De Groot et al.~\cite{DBLP:journals/iacr/GrootPPSB16} have verified experimentally the theorem for orders $d=1,2$ in the ARM Cortex-M4.
\subsection{Experimental Setup \& Evaluation}
The implementation and SCA evaluation is performed on a smartcard equipped with an 8-bit, AVR-based ATMega163 microcontroller\footnote{\url{http://www.atmel.com/images/doc1142.pdf}}. 
The device features a 4.4~MHz clock, 1024 bytes of SRAM and 17 Kbytes of Flash memory. 
The acquisition of power traces is carried out using the Riscure PowerTracer\footnote{\url{https://www.riscure.com/security-tools/hardware/power-tracer}} and the Picoscope~5203 oscilloscope. The sampling rate is set at 31.5~MSamples/sec and the only post-processing applied is signal alignment.

The evaluation of the \emph{actual} security order of a masking scheme is, in general, an open problem. We often face the \emph{limited attack scope}, i.e. a given attack may not be able to exploit the available leakage due to e.g. an unsuitable choice of intermediate values or an incorrect power model. To address this problem, generic side-channel distinguishers and extensive profiling techniques have been developed~\cite{DBLP:journals/joc/BatinaGPRSV11,DBLP:conf/cardis/WhitnallOM11,DBLP:conf/eurocrypt/StandaertMY09}. In this work, we opt for the
leakage detection methodology~\cite{tvla} which prioritizes leakage detection over leakage exploitation, speeding up certain evaluation aspects. In detail, we employ the random vs. fixed, non-specific, 1st-order t-test. We perform a random vs. fixed acquisition and obtain two distinct tracesets $S_{fixed}$ and $S_{random}$, under the same encryption key. The input plaintext for $S_{fixed}$ is set to a fixed value, while for $S_{random}$, the input is uniformly random. The implementation receives the fixed or random plaintext in a non-deterministic and randomly-interleaved way (as recommended by Schneider et al. ~\cite{DBLP:conf/ches/SchneiderM15}). Following the data acquisition, the 1st-order t-test will assess whether the two sets $S_{fixed},S_{random}$ stem from the same population, using the following statistical test.
\begin{equation}
\begin{split}
H_{null}: \;\; \mu_{fixed} = \mu_{random} \\
H_{alt}: \;\; \mu_{fixed} \neq \mu_{random}
\end{split}
\end{equation}
\begin{equation}
w = \frac {\mu_{fixed} - \mu_{random}} {\sqrt{ \frac{\sigma_{fixed}^2}{n} + \frac{\sigma_{random}^2}   {m}  }  } \qquad \upsilon = \frac { (\frac{\sigma_{fixed}^2} {n}   + \frac{\sigma_{random}^2} {m}) ^2  } {\frac{\sigma_{fixed}^4} {n^2(n-1)} + \frac{\sigma_{random}^4} {m^2(m-1)}  }
\end{equation}


Parameters $\mu_{x}$ and $\sigma_{x}^2$ are the estimated mean and variance of set $x$; 
$n$ and $m$ denote sizes of sets $S_{fixed}$ and $S_{random}$ respectively. The null hypothesis $H_{null}$ is rejected at a given level of significance $\alpha$ (often set to $0.99999$), if $\lvert w \lvert >  t_{\alpha/2,\upsilon}  $, where $t_{\alpha/2,\upsilon}$ is the value of the Student t distribution with $\upsilon$ degrees of freedom. In the evaluation context, rejecting $H_{null}$ implies leakage detection, i.e. potential evidence of an ineffective masking scheme.

In this paper, we will use the t-test as a detection tool w.r.t. ILA-breaching effects and their solutions (see Section 3). Still, we will also employ 1st-order CPA methods~\cite{DBLP:conf/ches/BrierCO04} in order to demonstrate the exploitability of such effects. In order to reduce the computational cost of the evaluation, we use the memoryless formulas suggested by Schneider et al.~\cite{DBLP:conf/ches/SchneiderM15} and the incremental approach for CPA by Botinelli et al.~\cite{DBLP:journals/iacr/BottinelliB15}.



\section{ILA-Breaching Effects}\label{sec:ila_effects}

In this section, we present $<>$ effects \todo{add value} identified in the ATMega163 microcontroller that breach ILA and pose a hazard to any masking scheme's security. Analytically, the effects below demonstrate that independent computations \emph{do not} necessairly lead to independent leakages and thus the order-reduction theorem can become applicable.

Every effect (Sections~\ref{overwrite},~\ref{mem_remnant} and~\ref{combined_leakage} $<>$) is described as a standalone, assembly-based scenario that manipulates two 4-bit shares $x_0$, $x_1$ originating from the sensitive, key-dependent, 4-bit value $x$, such that $x=x_0 \oplus x_1$. The shares $x_0$, $x_1$ are always manipulated in a theoretically sound manner, adhering to the masking scheme's requirements, i.e. we never combine the shares directly (e.g. via an exclusive-or instruciton \texttt{eor x0, x1}). 

For all the described scenarios, that should be theoretically sound, we show experimentally that ILA is not fulfilled by employing 1st-order, univariate techniques. Namely, we perform correlation-based analysis~\cite{DBLP:conf/ches/BrierCO04}, computing $\rho$ between the Hamming weight of the sensitive, key-dependent value $x$ and the experimentally acquired traceset. To maintain a wide attack scope, we also use the leakage detection methodology~\cite{tvla,DBLP:conf/ches/SchneiderM15} and compute the 1st-order, random vs. fixed t-test. 
 We conclude every scenario by suggesting possible solutions that enforce ILA and verify all the proposed solutions experimentally, using a large number of traces. Restating Balasch et al.~\cite{DBLP:conf/cardis/BalaschGGRS14}, as we are always limited by the traces at hand, we cannot rule out the existence of 1st-order leakages, yet we establish that their informativeness is limited compared to 2nd-order leakages in the target device. Note that extra care is taken in order to assess all effects independently, i.e. we use the suggested solutions so as to isolate the effect under discussion from the rest.

The effects under discussion can manifest in several data storage units (e.g. registers, SRAM/Flash memory cells, I/O buffers, etc.) and may relate to different instructions of the AVR ISA\footnote{\url{http://www.atmel.com/images/Atmel-0856-AVR-Instruction-Set-Manual.pdf}}, leading to a very large number of potential scenarios. In order to maintain a feasible scope, we limit our discussion to storage instructions and instructions that are often encountered in the context of cryptographic implementation, i.e. logical instructions and SRAM memory accesses.
\begin{subsection}{Overwrite Effect}\label{overwrite}

The overwrite effect is observable when a share gets overwritten by a different share from the same family. For instance, if share $x_0$ in a data storage unit (register, memory cell, etc.) gets overwritten by share $x_1$, then the power consumption correlates with the number of bits switched i.e. $x_0 \oplus x_1$. This effect was observed by Daemen et al.~\cite{noteonsca} and later revisited by Coron et al.~\cite{DBLP:conf/cosade/CoronGPRRV12}.\\
 Below, we address the most common situations in which overwriting arises during a cryptographic implementation. We perform two experiments: a register-based overwrite via the instruction \texttt{mov x0, x1}, and a memory-based overwrite via \texttt{st SRAM\_x0 ,  x1}. 
%Note that overwriting effects can arise due to other instructions and/or storage units. For example, \texttt{ld x0, SRAM\_x1} will cause a similar %effect and it is also subject to the operation remnant effect (see Section \ref{remnant_section}). Overwriting the I/O buffers or partial overwriting via %the \texttt{bld, bst} instructions can produce the same effect (see Appendix A).
\begin{figure}[H]
    \centering
\begin{subfigure}[b]{0.4\textwidth}
      \texttt{;share x0 in r17 \\
;share x1 in r23 \\
\textbf{mov r17,r23}\\
			}

        \caption{\scriptsize{Register overwrite experiment.}}

    \end{subfigure}
 \begin{subfigure}[b]{0.4\textwidth}
         \texttt{;share x0 in SRAM 0x0080 \\
;share x1 in r17  \\
ldi r27,0x00 \\
ldi r26,0x80 \\
st X,r17\\
}

        \caption{\scriptsize{Memory overwrite experiment.}}

    \end{subfigure}

 \begin{subfigure}[b]{0.47\textwidth}
        %\includegraphics[width=\textwidth]{reg_over_cpa.png}

        \caption{\scriptsize{Register overwrite, 1st-order CPA, HW model, 500 traces.}}

    \end{subfigure}
 \begin{subfigure}[b]{0.47\textwidth}
       %\includegraphics[width=\textwidth]{reg_over_t_an.png}

        \caption{\scriptsize{Register overwrite, 1st-order t-test, 5k random vs. 5k fixed.}}

    \end{subfigure}

     %add desired spacing between images, e. g. ~, \quad, \qquad, \hfill etc. 
      %(or a blank line to force the subfigure onto a new line)
    \begin{subfigure}[b]{0.47\textwidth}
        %\includegraphics[width=\textwidth]{memory_over_cpa.png}
        \caption{\scriptsize{Memory overwrite, 1st-order CPA, HW model, 65k traces.}}
        \label{fig:tiger}
    \end{subfigure}
    %add desired spacing between images, e. g. ~, \quad, \qquad, \hfill etc. 
    %(or a blank line to force the subfigure onto a new line)
    \begin{subfigure}[b]{0.47\textwidth}
        %\includegraphics[width=\textwidth]{memory_over_t_an.png}
        \caption{\scriptsize{Memory overwrite, 1st order t-test, 50k random vs. 50k fixed.}}

    \end{subfigure}
    \caption{Register/memory-based overwrite effects}\label{fig:mem}
\end{figure}
%Causing overwrites is also possible via load/store instructions from/to Flash memory , yet the high computational cost of accessing the Flash memory %makes it less common situation in cryptographic implementations (although it has exhibited potential for AVR and TI-based %microcontrollers~\cite{shuffled, feram}). 
We confirm that overwriting is indeed an ILA-breaching effect, manifesting both in registers and SRAM memory. Note that the exploitability of the effect varies according to the data storage unit: in ATMega163, register-based overwriting can be exploited with roughly 500 traces, while memory-based requires at least 40k traces. Preventing register/memory-based overwrites is straightforward: the corresponding register/cell needs to be cleared in advance. We confirm experimentally that this solution remains secure against a $<>$k \todo{add value} random vs. $<>$k fixed, 1st-order t-test (see Appendix A\todo{add appendix}).
\label{overwrite_section}
\end{subsection}

\begin{subsection}{Memory Remnant Effect} \label{mem_remnant}

The memory remnant effect is a type of leakage originating from consecutive SRAM accesses to different shares of the same family. For instance, assume that shares $x_0$, $x_1$ are stored in SRAM cells and get accessed sequentially. Naturally, the first access leaks share $x_0$ (value-based leakage), yet it also creates a ``remnant" of $x_0$. The second memory access will leak share $x_1$ (value-based leakage) \emph{and} the remnant $x_0$ concurrently, reducing the security order of the scheme.

We address this scenario with the two following experiments. First, we demonstrate how two consecutive SRAM accesses \texttt{ld rA, SRAM\_x0} and \texttt{ld rB, SRAM\_x1} produce the remnant effect. Second, we show how clearing the register and inserting a dummy SRAM access can remove the remnant.    
\begin{figure}[H]
    \centering
\begin{subfigure}[b]{0.4\textwidth}
      \texttt{;share x0 in 0x0080 \\
;share x1 in 0x0090 \\
ldi r27, 0x00\\
ldi r26, 0x80\\
ld r17, X\\
ldi r27, 0x00\\
ldi r26, 0x90\\
ld r20, X
			}

        \caption{\scriptsize{Memory remnant experiment.}}

    \end{subfigure}
\begin{subfigure}[b]{0.4\textwidth}
          \texttt{;share x0 in 0x0080 \\
;share x1 in 0x0090 \\
ldi r27, 0x00\\
ldi r26, 0x80\\
ld r17, X\\
ldi r17, 0x00\\ 
ldi r26, 0x85\\
ld r17, X\\
ldi r26, 0x90\\
ld r20, X
			}
\todo[inline]{should it be r17 in 'ldi r17, 0x00' and in the 2nd 'ld r17, X'?}
        \caption{\scriptsize{Clearing remnant experiment.}}

    \end{subfigure}


 \begin{subfigure}[b]{0.47\textwidth}
        %\includegraphics[width=\textwidth]{mem_rem_cpa.png}

        \caption{\scriptsize{Memory remnant effect,1st-order CPA, HW model, 500 traces.}}

    \end{subfigure}
 \begin{subfigure}[b]{0.47\textwidth}
      %\includegraphics[width=\textwidth]{mem_rem_t_an.png}

        \caption{\scriptsize{Memory remnant effect, 1st-order t-test, 5k random vs. 5k fixed.}}

    \end{subfigure}
 \begin{subfigure}[b]{0.47\textwidth}
        %\includegraphics[width=\textwidth]{clear_rem_cpa.png}

        \caption{\scriptsize{Clearing remnant effect,1st-order CPA, HW model, 100k traces.}}

    \end{subfigure}
 \begin{subfigure}[b]{0.47\textwidth}
       %\includegraphics[width=\textwidth]{clear_rem_t.png}

        \caption{\scriptsize{Clearing remnant effect, 1st-order t-test, 100k random vs. 100k fixed.}}

    \end{subfigure}

   
    \caption{Memory remnant effect}\label{fig:regleak}
\end{figure}
As shown above, consecutive SRAM accesses can potentially lead to ILA violations. Preventing this requires the clearing of the register and the insertion of dummy SRAM. Alternatively, the implementor could ensure that same-family shares are not accessed sequentially. Exploiting (in a univariate manner) the memory remnant effect in ATMega163 requires less than 500 traces with our setup. The \texttt{st} instruction produces a similar effect (see Appendix A\todo{add appendix}).
\end{subsection}

\begin{subsection}{Combined Leakage Effect}\label{combined_leakage}

\todo[inline]{should we call this thing 'neighbour leakage effect'?}
The combined leakage effect implies that accessing or processing the contents of a data storage unit will cause leakage in another unit as well. For example, assume that share $x_0$ is stored in register \texttt{rB} and share $x_1$ is being processed in register \texttt{rA}. Assume also that the registers \texttt{rA, rB} are subject to the combined leakage effect. Processing \texttt{rB} will produce a value-based leakage of $x_1$. At the same time, the combined leakage effect will cause \texttt{rA} to leak the value of $x_0$, resulting in the concurrent leakage of both shares, revealing the sensitive value $x$.

The following two experiments verify the combined leakage effect between registers \texttt{r2, r3}, i.e. that a share stored in \texttt{r2} leaks when manipulating \texttt{r2} \emph{or} \texttt{r3} and vice-versa.
\begin{figure}[H]
    \centering
\begin{subfigure}[b]{0.4\textwidth}
\texttt{;clear all registers\\
;store x0 in r2 \\
mov r0, r0\\
NOPs \\
mov r1, r1\\
NOPs \\
mov r2, r2\\
NOPs \\
mov r3, r3 \\
NOPs \\
...\\
mov r31, r31 }

        \caption{\scriptsize{Combined leakage \texttt{r2-r3}.}}

    \end{subfigure}
\begin{subfigure}[b]{0.4\textwidth}
\texttt{;clear all registers\\
;store x0 in r3 \\
mov r0, r0\\
NOPs \\
mov r1, r1\\
NOPs \\
mov r2, r2\\
NOPs \\
mov r3, r3 \\
NOPs \\
...\\
mov r31, r31 }

        \caption{\scriptsize{Combined leakage \texttt{r3-r2}.}}

    \end{subfigure}


 \begin{subfigure}[b]{0.47\textwidth}
        %\includegraphics[width=\textwidth]{corr_r23_an.png}

        \caption{\scriptsize{Correlation \texttt{r2-r3} $\rho(HW(x_0),traceset)$, 5k traces.}}

    \end{subfigure}
 \begin{subfigure}[b]{0.47\textwidth}
        %\includegraphics[width=\textwidth]{corr_r32_an.png}

        \caption{\scriptsize{Correlation \texttt{r3-r2} $\rho(HW(x_0),traceset)$, 5k traces.}}

    \end{subfigure}

   
    \caption{Combined leakage effect.}\label{fig:regleak}
\end{figure}


\end{subsection}
As shown above, we have identified a pair of data storage units (\texttt{r2,r3}) that exhibit the combined leakage effect. Note that in this case the effect is symmetrical, i.e. \texttt{r2} triggers \texttt{r3} and vice-versa. We run the same experiment in order to identify all possible combined leakages in the register file. The results are available in Appendix A\todo{add appendix}, matrix $R$. The combined leakage mostly affects neighboring registers, although exceptions exist, e.g. register \texttt{r0}. We did not identify a similar effect in SRAM memory cells, yet our experiments were limited to a small region of neighboring cells 

\begin{subsection}{Other Leakage Effects}
\todo[inline]{PENDING}

\end{subsection}


Summing up, we stress the following focal points regarding the ILA-breaching effects and their solutions:
\begin{itemize}
\item All identified effects are device-dependent, i.e. there is no hard guarantee that they are observable and reproducible in different AVR-based (or other instruction set) microcontrollers, let alone different architectures such as ARM, TI, PIC etc. Both intra-AVR and inter-architectural observability of the effects remains open.
\item The effects are often counter-intuitive when viewed in the assembly layer of abstraction. They originate from the hardware and/or the physical layer, thus can only be detected via experimental evaluation. Linking the assembly ILA-breaching effects to a particular hardware component or physical phenomenon is non-trivial~\cite{renauld,others}\todo{bib ref}, especially without knowledge of the underlying chip architecture and properties.
\item  Since the effect's detection requires experimental evaluation, different instructions or code arrangements can potentially lead to additional, unidentified ILA-breaching effects. Still, we maintain that it is possible to construct ``hardened" masked operations in ATMega163 by removing the identified effects (see Section \ref{sbox_section}). It remains open whether the suggested solutions are computationally optimal or more efficient clearing techniques can be identified.
\end{itemize}

The takeaway message of this section is that assembly-level soundness cannot enforce ILA and hence 1st-order security, due to the nature of the breaching effects. However, it is possible to acquire sufficient knowledge about effects and solutions in a particular device. These non-intuitive checks discussed above can be subsequently integrated into a code-checking tool which can identify such effects in assembly code.

\section{Detection Tool}\label{sec:tool}

Several tools that can help designers of cryptographic systems
were already suggested and discussed in literature.

SILK~\cite{DBLP:conf/acsac/Veshchikov14} presented in 2014
can be used to generate simulated traces based on C++ code, it allows
to generate sets of traces on early stages of development in order to test an implementation
against any attack. However, SILK works only with C++ high level source code and 
can not take into account reordering of instructions that is often used by compilers during
optimisation. Also, this tool does not detect flaws in implementations, 
it only allows to easily generate simulated traces that can be used for tests.

A tool based on formal verification 
was presented at EUROCRYPT in 2015~\cite{DBLP:conf/eurocrypt/BartheBDFGS15},
it can detect design flaws in masking schemes.
This tool can analyse programs written using EasyCrypt framework and its language,
it requires a designer to transform the original implementation (e.g., in assembly or C code)
to EasyCrypt. Unfortunately, errors cound potentially be introduced during this process
and there is no garantee that the programm written using EasyCrypt will be equivalent
to the programm in the original programming language, the the best of our knowledge
free automated tools that can transform C or assembly programms to EasyCrypt do not exist.
Moreover, this tool is not opensource and thus can not be used by any developer.

A simulation tool based tool that can be used to analyse masking implementations
was presented at FSE in 2016~\cite{DBLP:conf/fse/Reparaz16}.
It can be used with software and hardware implementations
and it requires only the high-level implementation source code such as C.
Due to this fact it can be blind to rearangements of opperations 
(which can lead to side-channel leakage) created by the compiler.
Up until now, the source code of this tool also remaines unavailable for general public.


\todo[inline]{Our tool}



\section{Hardened 1st-order Masked Sbox: A Case Study Using RECTANGLE}\label{sec:rectangle}


















\section{Conclusions} \label{sec:conclusions}

This work investigated the hazards in software masking, suggested a verification tool and established a secure, 1st-order masked Sbox implementation against 1st-order, univariate attacks. Still, several important questions for future work arise. We demonstrated that removing the ILA-breaching effects is feasible, yet identifying the best clearing mechanism and minimizing the overhead is a topic for further exploration. Similarly, the current work is limited in AVR ATMega163 and needs to be extended to different devices and platforms. It could be done by using ASCOLD tool as a base for this kind of work.  Moreover, higher-order evaluation techniques are still nascent and in this work we did not focus on 1st-order, yet multivariate attacks such as those that exploit horizontality~\cite{DBLP:conf/ches/BattistelloCPZ16}. In addition, note that the ILA effects are observable throughout an implementation. Not only the cipher-related operations but any manipulation of shares during I/O, RNG routines etc. can create hazards. Thus, there is need for effort towards a fully hardened implementation. Last but not least, we stress that the effects identified depend on the architecture and the physical layer, thus preventing them in the assembly layer is, in principle, less efficient and prone to errors. Future work can strive towards custom-made microcontrollers that enforce ILA in hardware. Ideally, such a microcontroller should be able to guarantee ILA without additional countermeasures such as threshold implementations~\cite{DBLP:conf/icics/NikovaRR06}.  

%\section{Acknowledgments}
%Do we have any?

\bibliography{research}

\end{document}
% end of file template.tex

