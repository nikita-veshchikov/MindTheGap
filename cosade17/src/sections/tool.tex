\section{Detection Tool}\label{sec:tool}

Several tools that can help designers of cryptographic systems
were already suggested and discussed in literature.

SILK~\cite{DBLP:conf/acsac/Veshchikov14} presented in 2014
can be used to generate simulated traces based on C++ code, it allows
to generate sets of traces on early stages of development in order to test an implementation
against any attack. However, SILK works only with C++ high level source code and 
can not take into account reordering of instructions that is often used by compilers during
optimisation. Also, this tool does not detect flaws in implementations, 
it only allows to easily generate simulated traces that can be used for tests.

A tool based on formal verification 
was presented at EUROCRYPT in 2015~\cite{DBLP:conf/eurocrypt/BartheBDFGS15},
it can detect design flaws in masking schemes.
This tool can analyse programs written using EasyCrypt framework and its language,
it requires a designer to transform the original implementation (e.g., in assembly or C code)
to EasyCrypt. Unfortunately, errors cound potentially be introduced during this process
and there is no garantee that the programm written using EasyCrypt will be equivalent
to the programm in the original programming language, the the best of our knowledge
free automated tools that can transform C or assembly programms to EasyCrypt do not exist.
Moreover, this tool is not opensource and thus can not be used by any developer.

A simulation tool based tool that can be used to analyse masking implementations
was presented at FSE in 2016~\cite{DBLP:conf/fse/Reparaz16}.
It can be used with software and hardware implementations
and it requires only the high-level implementation source code such as C.
Due to this fact it can be blind to rearangements of opperations 
(which can lead to side-channel leakage) created by the compiler.
Up until now, the source code of this tool also remaines unavailable for general public.


\todo[inline]{Our tool}


