\section{Introduction}
Nowadays, the explosive growth of the ``Internet of Things'' (IoT) is reshaping modern society, pervading its infrastructure and communications.
The rapid price drop in IoT components has transformed everyday products, enhancing them with network connectivity and information exchange capabilities. Amidst this new status quo, devices ranging from cheap sensors to expensive vehicles, are required to maintain a heightened level of theoretical and physical security.

For instance, side-channel attacks (SCA) allow adversaries to recover sensitive data, by observing and analyzing the
physical characteristics and emanations of a cryptographic implementation~\cite{DBLP:conf/crypto/KocherJJ99}. Such physical attacks motivated research towards countermeasures that perform noise amplification, thus hindering the adversary's recovery capabilities. A common choice for provably secure, noise-amplifying software countermeasure is masking~\cite{DBLP:conf/crypto/ChariJRR99,DBLP:conf/crypto/IshaiSW03}. Masking employs secret-sharing techniques that establish theoretical security against the value-based leakage model. Rephrasing, masking secures implementations against adversaries that can only extract information about the value being processed at a given time and cannot perceive any leakage from idle values at the same time. This underlying assumption is often referred to as the \emph{independent leakage assumption} (ILA)~\cite{DBLP:conf/eurocrypt/RenauldSVKF11}. Unfortunately, such a limited adversarial model is not applicable in many practical, software-based scenarios. For instance, devices often exhibit \emph{distance-based leakages}, which can reduce the security of the masking countermeasures~\cite{DBLP:conf/cardis/BalaschGGRS14, DBLP:journals/iacr/GrootPPSB16}. Likewise, coupling effects~\cite{DBLP:conf/eurocrypt/RenauldSVKF11} and glitches~\cite{DBLP:conf/ches/MangardS06} can pose similar security hazards.

This work attempts to bridge the gap between theory and practice in the masking countermeasure with the following threefold contribution. First, we investigate several effects that violate ILA in an ATMega163 microcontroller and subsequently we establish solutions that help mitigate these issues. Second, we use this knowledge in order to build an assembly-oriented tool that is capable of detecting ILA violations in AVR-based masked implementations. Third, assisted by the developed tool, we craft the first (to our knowledge) 1st-order masked implementation in ATMega163 that is capable of resisting 1st-order, univariate attacks. In other words, we enforce the ILA in order to severely limit the informativeness of 1st-order leakages, forcing the adversary to resort to 2nd-order attacks. As a proof of concept, we develop a ``hardened" 1st-order, ISW-based~\cite{DBLP:conf/crypto/IshaiSW03}, bitsliced Sbox for the RECTANGLE cipher~\cite{DBLP:journals/chinaf/ZhangBLR0V15}. The ``hardened'' implementation requires $<>$\todo{add number of cycles} clock cycles, a $<>$-fold increase compared to a naive 1st-order, ISW-based, bitsliced Sbox of the same cipher. 

The paper is organized as follows. In Section~\ref{sec:background}, we provide preliminaries w.r.t. masking, the experimental setup and the evaluation techniques we employ. In Section~\ref{sec:ila_effects} we offer a detailed description all the ILA-breaching effects that we have identified in ATMega163. Section~\ref{sec:tool} discusses the development of the assembly checking tool. Section 5 details the construction of a ``hardened'' RECTANGLE, 1st-order masked Sbox for ATMega163. We conclude and discuss future works in Section~\ref{sec:conclusions}.


